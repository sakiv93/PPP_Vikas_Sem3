\documentclass[12pt]{article}

\usepackage{amsmath}
\usepackage{amsfonts}
\usepackage{amssymb}
\usepackage{graphicx}
\usepackage{caption}
\usepackage{mathtools}
\usepackage{lipsum}
\usepackage{stackengine}
\usepackage{fancyhdr}
\usepackage{caption}
\usepackage{tikz}
\usetikzlibrary{shapes.geometric, arrows}
\usepackage{float}
\usepackage[a4paper,left=1in,right=1in,top=1in,bottom=1in,footskip=.25in]{geometry}
\usepackage{etoolbox}
\usepackage[nottoc]{tocbibind}
\usepackage{tabu}
\begin{document}
	\begin{titlepage}
		\centering
		
		\begin{figure}
			\begin{center}
				\includegraphics[scale=.2]{tubaf.pdf}  
			\end{center}
			
		\end{figure}
		
		
		
		%\includegraphics[width=0.15\textwidth]{download}\par\vspace{1cm}
		{\scshape\LARGE Technische Universit\"at Bergakademie Freiberg \par}
		\vspace{1cm}
		{\scshape\Large PERSONAL PROGRAMMING PROJECT\par}
		\vspace{1.5cm}
		{\huge\bfseries Implementation of Iso-geometric Analysis (IGA) for Piezoelectric Material \par}
		\vspace{2cm}
		{\scshape\Large Vikas Diddige\par}
		{\scshape\Large 64041\par}
		\vfill
		{\normalsize\ Supervised by\par}
		
		Dr.~ \textsc{Sergii Kozinov}
		
		\vfill
		
		% Bottom of the page
		{\large \today\par}
	\end{titlepage}
	
	\clearpage
	\textbf{\LARGE Abstract}\\ \newline
	\newline
	
	
	\newpage
	\clearpage
	\tableofcontents
	\clearpage
	
	\section{Introduction}
	Among all the numerical methods Finite Element Methods (FEM) are more popularly used to find approximate solutions of partial differential equations. FEM approximates the Computer Aided Drawing (CAD) geometry by discretizing it into smaller geometries called elements. Such geometrical approximations may create numerical errors and seriously effect the accuracy of the solution.
	Isogeometric analysis (IGA) is a technique to generate geometry using CAD concept of Non Uniform Rational B-Splines and analyse using its basis functions. !The pioneers of this technique are Tom Hughes and his group at The University of Texas at Austin!.
	
	
	\subsection{Advantages of IGA over FEA}
	\begin{description}
		\item[$\bullet$]   The exact representation of the geometry for analysis rules out the possibility of geometrical approximations.
		\item[$\bullet$]   A huge amount of time involved in finite element modelling can be avoided.
	\end{description}
	
	\section{Non Uniform Rational B-Splines }
	NURBS are very often used in computer-aided design(CAD), manufacturing (CAM) and engineering (CAE) due to its flexibility to represent complex geometries. NURBS curves and surfaces are considered as the generalization of B-Spline and Bezier curves and surfaces. A NURBS curve is defined by its order, control points and knot vectors.
	
	\subsection{Order }
	!The order of a NURBS curve defines the number of nearby control points that influence any given point on the curve. The curve is represented mathematically by a polynomial of degree one less than the order of the curve.(WIKI)! (Difference b/w degree and order)
	
	\subsection{Control points }
	!The control points determine the shape of the curve.[8] Typically, each point of the curve is computed by taking a weighted sum of a number of control points. The weight of each point varies according to the governing parameter. For a curve of degree d, the weight of any control point is only nonzero in d+1 intervals of the parameter space. Within those intervals, the weight changes according to a polynomial function (basis functions) of degree d. At the boundaries of the intervals, the basis functions go smoothly to zero, the smoothness being determined by the degree of the polynomial.(WIKI)!
	
	\subsection{Knot vector }
	The knot vector is a sequence of parameter values that determines where and how the control points affect the NURBS curve. The number of knots is always equal to the number of control points plus curve degree plus one (i.e. number of control points plus curve order). The knot vector divides the parametric space in the intervals mentioned before, usually referred to as knot spans. Each time the parameter value enters a new knot span, a new control point becomes active, while an old control point is discarded. It follows that the values in the knot vector should be in nondecreasing order, so (0, 0, 1, 2, 3, 3) is valid while (0, 0, 2, 1, 3, 3) is not.(WIKI)!
	\subsection{NURBS Curve and its properties }
	The $p^{th}$ degree NURBS curve is given by

	\begin{equation}
	C(u) = \frac{\sum_{i=0}^{n}N_{i,p}(u)w_{i}P_{i}}{\sum_{i=0}^{n}N_{i,p}(u)w_{i}}
	\end{equation}
	refer NURBS book and write more info
	\subsection{NURBS Surface and its properties }
	A NURBS surface with degree p in u direction and degree q in v direction is defined as
	\begin{equation}
	S(u) = \frac{\sum_{i=0}^{n}\sum_{i=0}^{m}N_{i,p}(u)N_{j,q}(v)w_{i,j}P_{i,j}}{\sum_{i=0}^{n}\sum_{i=0}^{m}N_{i,p}(u)N_{j,q}(v)w_{i,j}}
	\end{equation}
	refer NURBS book and write more info
	\section{Iso-geometric analysis}
	
	
	
	
	
\end{document}
