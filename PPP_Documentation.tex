\documentclass[11pt]{article}

\usepackage{amsmath}
\usepackage{amsfonts}
\usepackage{amssymb}
\usepackage{graphicx}
\usepackage{caption}
\usepackage{mathtools}
\usepackage{lipsum}
\usepackage{stackengine}
\usepackage{fancyhdr}
\usepackage{caption}
\usepackage{tikz}
\usetikzlibrary{shapes.geometric, arrows}
\usepackage{float}
\usepackage[a4paper,left=1in,right=1in,top=1in,bottom=1in,footskip=.25in]{geometry}
\usepackage{etoolbox}
\usepackage[nottoc]{tocbibind}
\usepackage{tabu}
\begin{document}
\begin{titlepage}
	\centering
	
	\begin{figure}
	\begin{center}
	\includegraphics[scale=.2]{tubaf.pdf}  
	\end{center}
	
	\end{figure}
	
	
	
	%\includegraphics[width=0.15\textwidth]{download}\par\vspace{1cm}
	{\scshape\LARGE Technische Universit\"at Bergakademie Freiberg \par}
	\vspace{1cm}
	{\scshape\Large PERSONAL PROGRAMMING PROJECT\par}
	\vspace{1.5cm}
	{\huge\bfseries Implementation of Iso-geometric Analysis (IGA) for Piezoelectric Material \par}
	\vspace{2cm}
	{\scshape\Large Vikas Diddige\par}
	{\scshape\Large 64041\par}
	\vfill
	{\normalsize\ Supervised by\par}
	
	Dr.~ \textsc{Sergii Kozinov}

	\vfill

% Bottom of the page
	{\large \today\par}
\end{titlepage}

\clearpage
\textbf{\LARGE Abstract}\\ \newline
\newline
                                                                                   

\newpage
\clearpage
\tableofcontents
\clearpage

\section{Introduction}
Among all the numerical methods Finite Element Methods (FEM) are more popularly used to find approximate solutions of partial differential equations. FEM approximates the Computer Aided Drawing (CAD) geometry by discretizing it into smaller geometries called elements. Such geometrical approximations may create numerical errors and seriously effect the accuracy of the solution.
Isogeometric analysis (IGA) is a technique to generate geometry using CAD concept of Non Uniform Rational B-Splines and analyse using its basis functions. !The pioneers of this technique are Tom Hughes and his group at The University of Texas at Austin!.


\section{Advantages of IGA over FEA}
\begin{description}
\item[$\bullet$]   The exact representation of the geometry for analysis rules out the possibility of geometrical approximations.
\item[$\bullet$]   A huge amount of time involved in finite element modelling can be avoided.
\end{description}

\section{Non Uniform Rational B-Splines }
NURBS are very often used in computer-aided design(CAD), manufacturing (CAM) and engineering (CAE) due to its flexibility to represent complex geometries. NURBS curves and surfaces are considered as the generalization of B-Spline and Bezier curves and surfaces. A NURBS curve is defined by its order, control points and knot vectors.

\subsection{Order }

\subsection{Control points }

\subsection{Knot vector }


\end{document}
