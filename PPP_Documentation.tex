\documentclass[11pt]{article}

\usepackage{amsmath}
\usepackage{amsfonts}
\usepackage{amssymb}
\usepackage{graphicx}
\usepackage[center]{caption}
\usepackage{mathtools}
\usepackage{lipsum}
\usepackage{stackengine}
\usepackage{fancyhdr}
\usepackage{caption}
\usepackage{tikz}
\usetikzlibrary{shapes.geometric, arrows}
\usepackage{float}
\usepackage[a4paper,left=1in,right=1in,top=1in,bottom=1in,footskip=.25in]{geometry}
\usepackage{etoolbox}
\usepackage[nottoc]{tocbibind}
\usepackage{tabu}
\usepackage{enumitem,kantlipsum}
\usepackage{verbatim}
\usepackage{hyperref}
\begin{document}
\begin{titlepage}
	\centering
	
	\begin{figure}
		\begin{center}
			\includegraphics[scale=.2]{tubaf.pdf}  
		\end{center}
		
	\end{figure}
	

	
	
	
	%\includegraphics[width=0.15\textwidth]{download}\par\vspace{1cm}
	{\scshape\LARGE Technische Universit\"at Bergakademie Freiberg \par}
	\vspace{1cm}
	{\scshape\Large PERSONAL PROGRAMMING PROJECT\par}
	\vspace{1.5cm}
	{\huge\bfseries Implementation of Iso-geometric Analysis (IGA) for
		Piezoelectric Material \par}
	\vspace{2cm}
	{\scshape\Large Vikas Diddige\par}
	{\scshape\Large 64041\par}
	\vfill
	{\normalsize\ Supervised by\par}
	
	Dr.~ \textsc{Sergii Kozinov}
	
	\vfill
	
	% Bottom of the page
	{\large \today\par}
\end{titlepage}

\clearpage

\newpage
\clearpage
\tableofcontents
\clearpage

\vspace*{1cm}

\section{Introduction}
Among all the numerical methods Finite Element Methods (FEM) are more popularly
used to find approximate solutions of partial differential equations. FEM
approximates the Computer Aided Drawing (CAD) geometry by discretizing it into
smaller geometries called elements. Such geometrical approximations may create
numerical errors and seriously effect the accuracy of the solution.
Isogeometric analysis (IGA) on the other hand, is a technique to generate geometry using CAD
concept of Non Uniform Rational B-Splines (NURBS) and analyse using its basis
functions \cite{agrawal2019iga}.
With the use of NURBS basis functions instead of Lagrangian basis functions the geometry is captured exactly for the analysis and rules out the possibility of the geometrical errors. Moreover, the time from design to analysis phase is greatly reduced saving the cost and time for the industry.
The IGA technique is firstly pioneered by Tom Hughes and his group at The
University of Texas at Austin.\\
\par
The present programming project aims at writing a python code to implement IGA for piezoelectric materials which involves electro-mechanical coupling. The coding work involves writing functions for 2D NURBS geometry generation, analysis with different load conditions and post processing of the results using contour plots. The pure mechanical case, electrical case and electro-mechanical coupling has been analysed and results are verified using abaqus results. Comparing the IGA program generated results with abaqus elements output is justified because of the fact that IGA aims at reducing the approximation from traditional FEM proceduers.      


\newpage


\subsection{Advantages of IGA over FEA}
\begin{description}[leftmargin=*]
	\item[$\bullet$]   The exact representation of the geometry for analysis rules
	out the possibility of geometrical approximations.
	\item[$\bullet$]   A huge amount of time and effort involved in finite element modelling
	can be avoided.
\end{description}

\section{B-Splines } \label{B_Spline}
In this section a brief description of B-Splines is discussed since NURBS is an
extended version of B-Splines. A B-Spline basis function is defined by its order
and knot vectors. A B-spline basis function along with control points defines a
B-Spline curve. A surface or volume can be generated using a curve by tensor product between its basis functions which will be discussed in detail in the future sections. 
\begin{figure}[H]
	\begin{center}
		\includegraphics[scale=0.8]{B_Spline.png} 
		\caption{\\A B-Spline curve with six control points}\label{B_Spline}
	\end{center}	
\end{figure}
\subsection{Order }
For a point on a B-Spline curve the order of the basis function speaks about the
number of nearby control points influence the given point. The degree $p$ of the
basis function is one less than the order of the curve.
The following figure shows a B-Spline curve with same number and position of control points but with different orders.
\begin{figure}[H]
	\begin{center}
		\includegraphics[scale=0.8]{DegreeBspline.png} 
		\caption{\\B-Spline curves with same control points but different order}\label{DegreeBspline}
	\end{center}	
\end{figure}

\subsection{Knot vector } \label{KnotVectorSection}
A knot vector is an array with an ascending order of parameter values written as
$\Xi = \{ \xi_0,\xi_1,\xi_2.....,\xi_{n+p}\}$
($\xi_i$ is called $ith$ knot and $\{\xi_i,\xi_{i+1}\}$ is called $ith$ Knot span), with $n+1$ basis functions which will be discussed in later sections.
The number of knots in a knot vector is equal to the summation of the degree of the curve and total number of control points defining the curve. 
B-Spline curves are defined in parametric space which is divided by knot spans. 
Knot vector should be an ascending order of knots. For example $\{0,0,1,1,2,3,3\}$ is valid but not $\{0,0,1,2,3,2,3\}$.
There is no difference between $\Xi = \{0,0,0,1,2,3,4,4,4\}$ and $\Xi = \{0,0,0,1/4,2/4,3/4,1,1,1\}$ which can be seen from Fig.(\ref{SameKnotVector}) because the latter can be obtained by dividing the former by 4.
\begin{figure}[H]
	\begin{center}
		\includegraphics[scale=0.8]{SameKnotVector.png} 
		\caption{\\B-Spline curves showing same trend with different Knot Vectors}\label{SameKnotVector}
	\end{center}	
\end{figure}
The effect of the control points on a B-Spline curve is completely defined by knot vector parameter values as shown in Fig.(\ref{DifferentKnotVector})
\begin{figure}[H]
	\begin{center}
		\includegraphics[scale=0.8]{DifferentKnotVector.png} 
		\caption{\\B-Spline curves with same control points but with different Knot vectors}\label{DifferentKnotVector}
	\end{center}	
\end{figure}

\subsection{Control points }
The co-ordinates and number of the control points determine the shape of the curve and the shape can be varied by altering the knot values in knot vector as discussed in section (\ref{KnotVectorSection}).
A span on the B-Spline curve is controlled by $p+1$ number of points.
The total
number of control points is given by $n_{cp}(\xi)$ = total number of knots in
$[\Xi] - (p+1)$. For an $pth$ degree curve atleast $p+1$ control points have to be defined.  


\subsection{B-Spline basis functions }

For a given Knot vector $\Xi$, the B-spline basis function for polynomial degree
$\geq 1$ is defined by a recursive function \cite{piegl2012nurbs}

\begin{equation}
N_{i,p}(\xi) = \frac{\xi-\xi_{i}}{\xi_{i+p}-\xi_{i}} N_{i,p-1}(\xi) + 
\frac{\xi_{i+p+1}-\xi}{\xi_{i+p+1}-\xi_{i+1}} N_{i+1,p-1}(\xi)
\end{equation}

\noindent
with
\begin{equation}
N_{i,0}(\xi) = 
\begin{cases*}
1 \quad& if $  \xi_{i} \leq \xi<\xi_{i+1} $ \\
0 &  $ otherwise $ \\
\end{cases*}
\end{equation}

\subsubsection{Properties }
\begin{enumerate}
	\item $ N_{i,0}(\xi)$ is a step wise function with a value 1 over the half open
	interval $ \xi \in [\xi_{i}  \leq \xi<\xi_{i+1}) $ and zero on the rest.
	\item Basis functions sum upto to unity $\sum_{i=0}^{n} N_{i,p}(\xi) =1$
	\item Basis functions are non-negative $ N_{i,p}(\xi) \geq 0$ over the entire
	domain
\end{enumerate}

\begin{figure}[H]
	\begin{center}
		\includegraphics[scale=0.7]{BSplineBasisFunctions.png}
		\caption{\\Cubic B-Spline functions with a uniform knot vector \cite{piegl2012nurbs}   }\label{BSplineBasisFunctions}
	\end{center}	
\end{figure}
\noindent
The following python function outputs non zero basis functions in a given knot vector span
\begin{verbatim}
-----------------------------------------------------------------------------------
def BasisFuns(i,u,p,U):
#----------Inputs---------#
i - knot span
u - parametric co-ordinate
p - Degree of the curve
U - knot vector of the curve
#----------Outputs---------#
non zero basis functions for the given parametric co-ordinate
-----------------------------------------------------------------------------------
\end{verbatim}

\subsubsection{Derivatives }
The first derivative of a B-Spline basis function \cite{nguyen2012introduction} with its variable $\xi$ is
given by

\begin{equation}
\frac{d}{d\xi}N_{i,p}(\xi) = \frac{p}{\xi_{i+p}-\xi_{i}} N_{i,p-1}(\xi) -
\frac{p}{\xi_{i+p+1}-\xi_{i+1}} N_{i+1,p-1}(\xi)
\end{equation}

\noindent
Higher order derivatives are not necessary for IGA formulation.\\
The following python function outputs derivatives of non zero basis functions in a given knot vector span.
\begin{verbatim}
-----------------------------------------------------------------------------------
def DersBasisFuns(i,u,p,m,U):
#----------Inputs---------#
i - knot span
u - parametric co-ordinate
p - Degree of the curve
U - knot vector of the curve
m - derivatives upto and including m(th) 
#----------Outputs---------#
The function returns non zero basis functions and their derivatives
upto and including mth derivative for the given parametric co-ordinate
-----------------------------------------------------------------------------------
\end{verbatim}

\subsection{B-Spline curves}

A $pth-degree$ B-Spline curve with a set of control points $P_i$ is given by \cite{piegl2012nurbs}
\begin{equation}
C(\xi) = \sum_{i=0}^{n} N_{i,p}(\xi) P_i \qquad \xi_0 \leq \xi \leq \xi_{n+p}
\end{equation}

\noindent
defined on the knot vector $\Xi = \{ \xi_0,\xi_1,\xi_2.....,\xi_{n+p}\}$

\subsection{B-Spline surfaces}
A B-Spline surface $S(\xi,\eta)$ is built by tensor product between B-Spline
curves along each parametric direction ($\xi,\eta$). It requires knot vectors
($\Xi = \{ \xi_0,\xi_1,\xi_2.....,\xi_{n+p}\}$,$H =
\{\eta_0,\eta_1,\eta_2.....,\eta_{m+q}\}$) along each parametric direction and
control net $P_{i,j}$
\begin{figure}[H]
	\begin{center}
		\includegraphics[scale=0.7]{B_Spline_Surface.png} 
		\caption{\\A B-Spline surface with control points net \cite{siggel2017tigl}}\label{B_Spline_Surface}
	\end{center}	
\end{figure}


\begin{equation} \label{BSplineSurface}
S(\xi,\eta) = \sum_{i=0}^{n}\sum_{j=0}^{m} N_{i,p}(\xi) N_{j,q}(\eta) P_{i,j}
\end{equation}

\noindent
where
p,q are the degrees of the B-Spline basis functions along $\xi,\eta$
respectively and n,m are number of control points along $\xi,\eta$ respectively.
\noindent
Eq. (\ref{BSplineSurface}) can be compactly written as


\begin{equation} \label{BSplineSurface}
S(\xi,\eta) = \sum_{i=0}^{n}\sum_{j=0}^{m} N_{i,j}^{p,q}(\xi,\eta) P_{i,j}
\end{equation}

\subsubsection{Derivatives}

The partial derivatives of bivariate B-Spline basis functions w.r.t parametric
co-ordinates is given as \cite{nguyen2012introduction}

\begin{equation} 
\frac{\partial N_{i,j}^{p,q}(\xi,\eta)}{\partial \xi} = \frac{d}{d\xi}
\bigg(N_{i,p}(\xi)\bigg)N_{j,q}(\eta) 
\qquad
\frac{\partial N_{i,j}^{p,q}(\xi,\eta)}{\partial \eta} = \frac{d}{d\eta}
\bigg(N_{j,q}(\eta)\bigg)N_{i,p}(\xi)
\end{equation}


\section{Non Uniform Rational B-Splines } \label{NURBS}
NURBS are very often used in computer-aided design(CAD), manufacturing (CAM) and
engineering (CAE) due to its flexibility to represent complex geometries. NURBS
curves and surfaces are considered as the generalization of B-Spline and Bezier
curves and surfaces. A NURBS basis function is defined by its order and knot
vector.

\subsection{NURBS basis functions}
NURBS basis functions $R_{i,p}(\xi)$ are defined as \cite{piegl2012nurbs}
\begin{equation}\label{NURBSBasisFuncs}
R_{i,p}(\xi) = \frac{N_{i,p}(\xi)w_{i}}{\sum_{i=0}^{n}N_{i,p}(\xi)w_{i}}
\end{equation}

\noindent
where $N_{i,p}(\xi)$ is the $i$th B-Spline basis function with degree p and
$w_{i}$ denotes weight of the $i$th control point ($P_i$). ***** Explain weights
with help of an example****. When $w_{i} = constant \quad \forall i$ the NURBS
basis function reduces to B-Spline basis function.
\\The usage of weights can be illustrated in Fig.(\ref{WeightsCircles}). The same circle can be drawn with different number of control points by altering their weights.
\begin{figure}[H]
	\begin{center}
		\includegraphics[scale=1.0]{WeightsCircles.png} 
		\caption{\\NURBS circle with different control points and weights *** Refer This}\label{WeightsCircles}
	\end{center}	
\end{figure}

\subsubsection{Derivatives}
The first derivative of a NURBS basis function with its variable $\xi$ is given
by \cite{nguyen2012introduction}

\begin{equation}
\frac{d }{d \xi} R_{i,p}(\xi) = \frac{N^{'}_{i,p}(\xi)  W(\xi) - N_{i,p}(\xi) 
	W^{'}(\xi)}{W^{2}(\xi)}w_{i}
\end{equation}
\noindent
where
$N^{'}_{i,p}(\xi) = \frac{d }{d \xi} N_{i,p}(\xi)$  \\
\\
\noindent
and $W^{'}(\xi) =  \sum_{i=0}^{n}N^{'}_{i,p}(\xi) w_i$


\subsection{NURBS Curves }
The $p^{th}$ degree NURBS curve is given by \cite{piegl2012nurbs}

\begin{equation}
C(\xi) =
\frac{\sum_{i=0}^{n}N_{i,p}(\xi)w_{i}P_{i}}{\sum_{i=0}^{n}N_{i,p}(\xi)w_{i}}
\qquad \xi_0 \leq \xi \leq \xi_{n+p}
\end{equation}
in short form
\begin{equation}
C(\xi) = \sum_{i=0}^{n}R_{i,p}(\xi)P_{i}
\end{equation}
\\
A NURBS curve with different weights on control points along with its basis function is shown in Fig.(\ref{NURBSCurveBasisFunctions})
\begin{figure}[H]
	\begin{center}
		\includegraphics[scale=0.7]{NURBSCurveBasisFunctions.png} 
		\caption{\\$\Xi=\{0,0,0,0,1,2,3,4,4,4,4\}, w = \{1,1,1,3,1,1,1\}$\\(a) A third degree NURBS curve; (b) Associated NURBS basis functions \cite{piegl2012nurbs}}\label{NURBSCurveBasisFunctions}
	\end{center}	
\end{figure}
\subsubsection{Properties }
\begin{enumerate}
	\item NURBS basis functions sum upto to unity $\sum_{i=0}^{n} R_{i,p}(\xi) =1$
	\item NURBS basis functions are non-negative $ R_{i,p}(\xi) \geq 0$ over the entire
	domain
	\item $R_{0,p}(0)=R_{n,p}(1)=1$
	\item For $w_i=1$ for all $i$, NURBS basis functions $R_i(\xi)$ reduces B-Spline basis functions $N_i(\xi)$ 
\end{enumerate}
\subsection{NURBS Surfaces and solids }
NURBS Surfaces and solids are generated by the tensor product between NURBS
curve basis functions.
\begin{enumerate}[leftmargin=*]
	\item NURBS Surfaces: \\
	A NURBS surface with degree $p$ in $\xi$ direction and degree $q$ in $\eta$
	direction is defined as \cite{agrawal2019iga}
	\begin{equation}
	S(\xi,\eta) = \sum_{i=0}^{n}\sum_{j=0}^{m} R_{i,j}^{p,q}(\xi,\eta)  P_{i,j}
	\end{equation}
	where the bivariate NURBS basis functions are given by
	\begin{equation}
	R_{i,j}^{p,q}(\xi,\eta)  =
	\frac{N_{i,p}(\xi)N_{j,q}(\eta)w_{i,j}}{\sum_{i=0}^{n}\sum_{j=0}^{m}N_{i,p}(\xi)N_{j,q}(\eta)w_{i,j}}
	\end{equation}
	\item NURBS Solids: \\
	A NURBS solid with degree $p,q,r$ in $\xi$, $\eta$, $\zeta$ directions
	respectively is defined as \cite{agrawal2019iga}
	\begin{equation}
	S(\xi,\eta,\zeta) = \sum_{i=0}^{n}\sum_{j=0}^{m}\sum_{k=0}^{l}
	R_{i,j,k}^{p,q,r}(\xi,\eta,\zeta)  P_{i,j,k}
	\end{equation}
	where the $R_{i,j,k}^{p,q,r}(\xi,\eta,\zeta)$ is given by
	\begin{equation}
	R_{i,j,k}^{p,q,r}(\xi,\eta,\zeta)  =
	\frac{N_{i,p}(\xi)N_{j,q}(\eta)N_{k,r}(\zeta)w_{i,j,k}}{\sum_{i=0}^{n}\sum_{j=0}^{m}
		\sum_{k=0}^{l} N_{i,p}(\xi)N_{j,q}(\eta)N_{k,r}(\zeta)w_{i,j,k}}
	\end{equation}
\end{enumerate}

\subsection{Derivatives of NURBS bivariate Basis Functions}
The first partial derivatives of NURBS bivariate basis function are given by \cite{nguyen2012introduction}

\begin{equation}
\frac{\partial R_{i,j}^{p,q}}{\partial \xi} = \frac{N^{'}_{i,p} N_{j,q} W -
	N_{i,p} N_{j,q} W^{'}_{\xi}}{W^2}w_{i,j}
\end{equation}

\begin{equation}
\frac{\partial R_{i,j}^{p,q}}{\partial \eta} = \frac{N_{i,p} N^{'}_{j,q} W -
	N_{i,p} N_{j,q} W^{'}_{\eta}}{W^2}w_{i,j}
\end{equation}

\noindent
where

\begin{equation}
W = \sum_{i=0}^{n}\sum_{j=0}^{m}N_{i,p} N_{j,q}w_{i,j}
\end{equation}

\begin{equation}
W^{'}_{\xi} = \sum_{i=0}^{n}\sum_{j=0}^{m}N^{'}_{i,p} N_{j,q}w_{i,j}
\end{equation}

\begin{equation}
W^{'}_{\eta} = \sum_{i=0}^{n}\sum_{j=0}^{m}N_{i,p} N^{'}_{j,q}w_{i,j}
\end{equation}
The first partial derivatives of the trivariate basis functions can be computed
in a similar manner as bivariate basis functions using a chain rule.



\section{Implementation Procedure for IGA}
This section describes step-by-step implementation of the Isogeometric analysis.
A modified FEM code structure can be used to implement IGA. Similar to FEM, IGA
can be divided into pre-processing, processing and post-processing stages. The
flow in an IGA analysis in each stage is shown with the help of an flow chart in respective sections.

\begin{figure}[H]
	\begin{center}
		\includegraphics[scale=0.4]{PreProcessing.png} 
		\caption{\\Flow chart describing Pre-processing Stage of IGA}\label{PreProcessing}
	\end{center}
	
\end{figure}
\subsection{Pre-processing Stage of the Analysis}
This subsection mainly deals with NURBS based geometry creation, types of
assembly arrays needed to assemble discretized geometries and how to deal with
homogeneous and non-homogeneous boundary conditions on boundary defining control
points due to their higher ($C^{p-1}$) continuity unlike $C^{0}$ continuity of
FEM nodes. 

\subsubsection{Geometry Creation} \label{GeometryCreation}
As mentioned before in the section \ref{NURBS},apart from the physical details of the geometry like length, width, thickness etc. the construction of NURBS
discretized geometry requires parametric details such as control points, knot
vectors and the order of the NURBS curve. Commercial softwares like "Rhino" can
be used to extract parametric details of the complex NURBS geometry. \\
The following function in python is used for finding a point on NURBS surface.
\begin{verbatim}
NURBS_Surface_Point(n,p,U,m,q,V,Pw,u,v)
\end{verbatim}
Where, p and q are degree of the NURBS curve in $\xi$ and $\eta$ directions respectively, Pw control points vector, U and V are knot vector in $\xi$ and $\eta$ directions respectively. 
n and m are calculated using below code
\begin{verbatim}
n=(np.size(U)-1)-p-1
m=(np.size(V)-1)-q-1 
\end{verbatim}
u and v are any points in knot span, for example if 
$U = \{0,0,1,2,3,4,4\}$,
u can be 2.5 which belongs to knot span $\{2,3\}$ in U.\\
The entire surface can be generated by looping over knot span points and by extracting surface points by using above function and plotting the output. 
An example output of a geometry is shown below
\begin{figure}[H]
	\begin{center}
		\includegraphics[scale=0.6]{Geometry.png} 
		\caption{\\NURBS Geometry}\label{Geometry}
	\end{center}	
\end{figure}
Parametric details of the geometry \cite{agrawal2019iga}:
\begin{verbatim}
Pw=
[[-1.,0.,0.,1],[-0.85,0.35,0.,0.85],[-0.35,0.85,0.,0.85],[0.,1.,0.,1]],
[[-2.5,0.,0.,1.],[-2.5,0.75,0.,1.],[-0.75,2.5,0.,1.],[0.,2.5,0.,1.]],
[[-4,0.,0.,1.],[-4,4.,0.,1.],[-4.,4.,0.,1.],[0.,4.,0.,1.]]
p=2 and q=2 # Degree of the curve in xi and eta directions
U = [0., 0., 0., 1., 2., 2., 2.]
V = [0., 0., 0., 1., 1., 1.]
\end{verbatim}


 

\subsubsection{Assembly arrays}
Assembly arrays are required to assemble local discretized geometries to global
geometry. For IGA as knot vector and control points defines the geometry two
assembly arrays (1) Control point assembly array and (2) Knot vector
connectivity array are required.
\begin{enumerate}[leftmargin=*]
	\item Control point assembly array: \\
	The degree of the NURBS curve determines the number of control points
	($n^e_{cp}$) present in an IGA element.
	Considering a two-dimensional element ($\Omega^e$)\\
	$n^e_{cp}$ = $(p+1)(q+1)$ \\
	The details of the control points are stored row wise in assembly array.
	The following function generates control point assembly array
	\begin{verbatim}
	def ControlPointAssembly(n,p,m,q,ele_no):
	
	#-----------------Inputs-----------------------#
	# n-no.of control points along xi direction
	# p-Degree of basis function along xi direction
	# m-no.of control points along eta direction
	# q-Degree  of basis function along eta direction
	#-----------------Output-----------------------#
	CP - Control point assembly array
	\end{verbatim}
	A control point assembly contains a two dimensional array with control point numbers of each element, row wise.
	An assembly array with 2 elements as in Fig.(\ref{Element}) is shown below
	\begin{equation} \label{CPassembly}
	\textbf{CP} =
	\begin{bmatrix}
		1 & 2 & 4 & 5 \\
		2 & 3 & 5 & 6 \\
	\end{bmatrix}
	\end{equation}
	\begin{figure}[H]
		\begin{center}
			\includegraphics[scale=0.2]{Element.png} 
			\caption{\\A simple 2D square geometry with two elements}\label{Element}
		\end{center}	
	\end{figure}
	\item Knot vector connectivity array: \\
	The knot vector connectivity matrix \textbf{CP} is a row wise matrix with each row
	corresponds to respective element. The columns corresponds to span
	ranges (Span$\_$U and Span$\_$V) along $\xi$ and $\eta$ directions. \\
	Knot vector array along $\xi$ and $\eta$ direction for geometry, Fig.(\ref{Element}): \\
	$U = [ 0, 0, 1, 2, 2 ]$ \\
	$V = [ 0, 0, 1, 1 ]$ \\
	Span$\_$U is a 2D array which has rows equal to number of elements in $\xi$ direction.
	\begin{equation} \label{SpanU}
	\textbf{Span\_U} =
	\begin{bmatrix}
	0 & 1  \\
	1 & 2  \\
	\end{bmatrix}
	\end{equation}
	Similarly with only one element along $\eta$ direction
	\begin{equation} \label{SpanU}
	\textbf{Span\_V} =
	\begin{bmatrix}
	0 & 1  \\
	\end{bmatrix}
	\end{equation}
	Knot connectivity for above geometry is given by,
	\begin{equation} \label{CP}
	\textbf{CP} =
	\begin{bmatrix}
	1 & 1  \\
	2 & 1  \\
	\end{bmatrix}
	\end{equation}
\\
	The following lines of code executes Knot Connectivity array
	\begin{verbatim}
	uniqueU = np.unique(U)
	uniqueV = np.unique(V)
	nelU = ncpxi-p
	nelV = ncpeta-q
	CP = np.zeros((nel,2)) 
	Span_U=np.zeros((nelU,2))#Span_U have rows equal to number of elements in xi direction
	Span_V=np.zeros((nelV,2))#Span_V have rows equal to number of elements in eta direction
	count=0
	for i in range(0,nelV):
	for j in range(0,nelU):
	CP[count,:] = [j+1,i+1] 
	# First and second coloumns of 'knotConnectivity' store the global number of knot 
	span ranges
	# or row number of Span_U and Span_V arrays
	Span_U[j,:]= [uniqueU[j],uniqueU[j+1]]
	Span_V[i,:]= [uniqueV[i],uniqueV[i+1]]
	count=count+1
	CP = knotConnectivity -1
	CP = CP.astype(int)
	\end{verbatim} 
	
	
	\begin{verbatim}

	\end{verbatim}
\end{enumerate}
\subsubsection{Boundary Conditions}
A brief description on how to define boundary conditions (BCS) is described in this section. When the number of control points in each element in each direction equals the degree of the curve plus one in respective direction, a traditional way of defining homogeneous and inhomogeneous boundary conditions as in FEM can be followed. In any other case a special treatment for defining BCS have to be followed which is not in the scope of this project. Procedures like least square minimization method are usually adopted for this purpose. 


\subsection{Processing Stage of the Analysis}
In the processing stage of analysis it is required to compute global elemental
stiffness matrix and global force vector and solve these for the solution field.
To formulate these matrices, it requires NURBS basis functions and their
derivatives evaluation. Numerical integration scheme like Gauss-Legendre rule is
employed to solve the volume and area integrals involved in forming the
stiffness matrix and internal force vector. Numerical integration involves
mapping elements from physical space to master space (which is also called unit
domain). As NURBS basis functions are defined in parametric space as given in
Eq. (\ref{NURBSBasisFuncs})it requires an additional mapping from physical space to parametric space
($\Omega_e \rightarrow \widetilde{\Omega_e}$). Later parametric space can be
mapped on to master space (($\widetilde{\Omega_e} \rightarrow
\overline{\Omega_e}$)). This procedure is illustrated in
Fig.(\ref{MasterPhysical}).
\begin{figure}[H]
	\begin{center}
		\includegraphics[scale=0.3]{Physical_Master.png} 
		\caption{\\Mapping IGA Physical element to Master
			element \cite{agrawal2019iga}}\label{MasterPhysical}
	\end{center}	
\end{figure}


\begin{enumerate}[leftmargin=*]
	\item Mapping from master space to parametric space \cite{agrawal2019iga}: \\
	Consider a discretized IGA surface which is defined in parametric space
	$\widetilde{\Omega_e}=[\xi_i,\xi_{i+1}]\otimes[\eta_i,\eta_{i+1}]$, Refer
	Fig.(\ref{MasterPhysical}). The NURBS basis functions and their derivatives are evaluated at
	$\xi,\eta$ of the element $\widetilde{\Omega_e}$. These $\xi,\eta$ co-ordinate
	values are calculated by a linear mapping as shown below
	\begin{equation}
	\xi=\frac{1}{2}[(\xi_{i+1}-\xi_{i})\overline{\xi}+(\xi_{i+1}+\xi_{i})]
	\end{equation} 
	\begin{equation}
	\eta=\frac{1}{2}[(\eta_{i+1}-\eta_{i})\overline{\eta}+(\eta_{i+1}+\eta_{i})]
	\end{equation}
	where $\overline{\xi},\overline{\eta}$ are the integration points defined in
	master space \\
	$\textbf{J}_2$ matrix is defined as 
	\begin{equation}
	\textbf{J}_2 = \frac{\partial\xi}{\partial \bar{\xi}}
	\frac{\partial\eta}{\partial \bar{\eta}}
	\end{equation} 
	Determinant of $\textbf{J}_2$ matrix is required in numerical integration
	scheme for linear mapping 
	
	
	\item Mapping from physical space to parametric space \cite{agrawal2019iga}: \\
	The Jacobian matrix $\textbf{J}_1$ used to map from physical space to
	parametric space ($\Omega_e \rightarrow \widetilde{\Omega_e}$) is computed as:
	
	\begin{equation} \label{J1Matrix1}
	\textbf{J}_1 =
	\large
	\begin{bmatrix}
	\frac{\partial x}{\partial \xi} &\frac{\partial x}{\partial \eta} \\
	\frac{\partial y}{\partial \xi} &\frac{\partial y}{\partial \eta} \\
	\end{bmatrix} 
	\end{equation}
	
	The components of the $\textbf{J}_1$ matrix are calculated using Eq.
	(\ref{Co-ordinate}).
	
	\begin{equation}
	\frac{\partial x}{\partial \xi} = \sum_{k=1}^{n_{cp}^e} \frac{\partial
		\textbf{R}_k}{\partial \xi} x_i
	\qquad
	\frac{\partial x}{\partial \eta} = \sum_{k=1}^{n_{cp}^e} \frac{\partial
		\textbf{R}_k}{\partial \eta} x_i
	\end{equation} 
	
	\begin{equation}
	\frac{\partial y}{\partial \xi} = \sum_{k=1}^{n_{cp}^e} \frac{\partial
		\textbf{R}_k}{\partial \xi} y_i
	\qquad
	\frac{\partial y}{\partial \eta} = \sum_{k=1}^{n_{cp}^e} \frac{\partial
		\textbf{R}_k}{\partial \eta} y_i
	\end{equation}
	
\end{enumerate}
The two different mappings described above can be illustrated with an example
considering $ \textbf{F}(x,y)$ integrated over the physical space $\Omega$

\begin{equation*}
\begin{split}
\int_{\Omega} \textbf{F}(x,y)d\Omega & = \sum_{e=1}^{nel} \int_{\Omega_e}
\textbf{F}(x,y) d\Omega  \\
& = \sum_{e=1}^{nel} \int_{\widetilde{\Omega_e}}
\textbf{F}(\xi,\eta)|\textbf{J}_1| d\xi d\eta \\
& = \sum_{e=1}^{nel} \int_{\overline{\Omega_e}}
\textbf{F}(\overline\xi,\overline\eta)|\textbf{J}_1||\textbf{J}_2| d\overline\xi
d\overline\eta  \\
& = \sum_{e=1}^{nel} \int_{-1}^{1} \int_{-1}^{1}
\textbf{F}(\overline\xi,\overline\eta)|\textbf{J}_1||\textbf{J}_2| d\overline\xi
d\overline\eta  \\
& = \sum_{e=1}^{nel} \left[ \sum_{i=1}^{n_{gp}^e}
\textbf{F}(\overline\xi_i,\overline\eta_i) gw_i |\textbf{J}_1||\textbf{J}_2|
d\overline\xi d\overline\eta \right]
\end{split}
\end{equation*}

\noindent
where $nel$ is the total number of elements and $n_{gp}^e$, $gw_i$ denotes the
number of Gauss points and their respective Gauss weights.\\
\noindent
The following function is used to get the J1 and J2 matrix along with their determinants.
\begin{verbatim}
--------------------------------------------------------------
def Jacobian12(xi,eta,elU,elV):
#-----------------Input------------------#
Parametric Co-ordinates (xi,eta) of Gauss Points
Knot span range (elU,elV) in xi and eta direction
#-----------------Output-----------------#
J1 and J2 matrices along with their determinants
--------------------------------------------------------------
\end{verbatim}
The formulated global stiffness matrix and force vector are solved using
numerical scheme like Newton-Raphson method for the solution field.
\\A flow chart describing processing stage flow is shown in Fig.(\ref{Processing})
\begin{figure}[H]
	\begin{center}
		\includegraphics[scale=1.0]{Processing.png} 
		\caption{\\Flow chart describing Processing Stage of IGA}\label{Processing}
	\end{center}
\end{figure}

\subsection{Post-processing Stage of the Analysis}
This section deals with visualization of the deformed geometry and how a
displacement and solution dependent variables are plotted. 

\begin{enumerate}[leftmargin=*]
	\item Visualization of the deformed NURBS geometry:
	
	Visualization of the deformed geometry will be done in the same manner as the
	visualization of the initial geometry. After determining the displacement field
	at control points they are added to the initial control points co-ordinates
	$[\textbf{P}]$.\\
	\\
	$[\textbf{P}_{new}]=[\textbf{P}]+[\textbf{U}]$ \\
	\\
	$[\textbf{P}_{new}]$ control points array is used to plot the deformed
	geometry.
	As discussed in section (\ref{GeometryCreation}), instead of initial control point matrix of the geometry \textbf{P}, \textbf{$P_{new}$} is given as input to the function 
	\begin{verbatim}
	def NURBS_Surface_Point(n,p,U,m,q,V,P_new,u,v):
	\end{verbatim}
	and the deformed geometry can be plotted.
	 
	
	\item Plotting of displacements and solution dependent variables:
	
	A contour plot can be made using the displacement values ($U$) on deformed or undeformed geometry.
	Due to the higher continuity of the NURBS basis functions stress recovery techniques are not necessary to extract solution field on the deformed geometry.  
	
	\textbf{plt.contourf} a matplotlib function is used to plot displacements, reaction forces, and electrical potentials, so that the comparison with abaqus post processing plots would be easier. The latter plots these solutions fields as a contour plot.
	
\end{enumerate}

\begin{figure}[H]
	\begin{center}
		\includegraphics[scale=0.5]{PostProcessing.png} 
		\caption{\\Flow chart describing Post-processing Stage of IGA}\label{PostProcessing}
	\end{center}
	
\end{figure}


\section{Mechanical Case}


\subsection{Governing Equations}
The governing equation for mechanical deformation is based on conservation of
linear momentum which can be written as
\begin{equation}\label{ConservationOfLM}
\sigma_{ij,i} + b_j = 0
\end{equation}
where $ \sigma_{ij} $ and $ b_i $ is the Cauchy stress tensor and the body force.
Due to the static nature of the analysis the inertial term is not included in
the eq(\ref{ConservationOfLM}).\\*
Stress and strain are related by following constitutive equation
\begin{equation}
\sigma_{ij} = c_{ijkl} \epsilon_{kl}
\end{equation}
The infinitesimal strain theory is adopted for the analyis in which
displacements of the material particles are considered to be very small compared
to the dimentions of the body under loading. Strain in a small strain setting
can be written as
\begin{equation}
\epsilon_{ij}=\frac{1}{2}[u_{i,j}+u_{j,i}]
\end{equation}
where $ u_{i} $ are the displacements in the body


\subsection{Weak Formulation}
Consider a domain $\Omega$ with $\Gamma_u$ as prescribed displacements and
$\Gamma_t$ as traction boundary conditions. The domain boundary can be
represented as $\Gamma = \Gamma_u \cup \Gamma_t$ and $\Gamma_u \cap \Gamma_t =
\Phi$
By using the principle of virtual work the eq(\ref{ConservationOfLM}) can be written as
\begin{equation}
\delta W = \int_\Omega (\sigma_{ij,i} + b_j ) \delta u_j dV = 0
\end{equation} 
with,
$u = u_o$ on $\Gamma_u$ (essential boundary condition) and
$\sigma_{ij}n_j = \bar{t}_j$ on $\Gamma_t$ (natural boundary condition) \\*
where $\delta u_j$ is the virtual displacement field, $n_j$ is unit normal to
the surface \\*
Applying integration by parts to the stress term under integral and by making
using of conservation of angular momentum ($ \sigma_{ij} = \sigma_{ji} $) and
Gauss divergence theorm (converting volume integral to surface integral) we
approach at the following equation
\begin{equation} \label{FinalWeakform}
\delta W = \int_{\Omega} \sigma_{ij} \epsilon_{ij} d\Omega - \left[
\int_{\Gamma} \bar{t}_j \delta u_j d\Gamma_t  + \int_{\Omega} b_j \delta u_j
d\Omega \right]
\end{equation}
as an additional requirement $\delta u_j$ must be zero at essential boundary
conditions ($\Gamma_u$) for a unique solution.


\subsection{IGA Formulation}
The advantage of IGA over FEM formulation lies in its basis functions
incorporation and its ability to capture the exact geometry. While the FEM uses
lagrangian basis functions, IGA uses NURBS basis functions which are used to
generate the geometry itself. As discussed in the previous sections a
multidimentional NURBS basis function is represented by
$R_{i,j,k}^{p,q,r}(\xi,\eta,\zeta) \rightarrow R_i$. The isogeometric element 
is represented by basis function $R_i$ and control points $P_i$ as \cite{agrawal2019iga}
\begin{equation} \label{Co-ordinate}
\textbf{x}^e = \sum_{i=1}^{n_{cp}^e} R_i P_i
\end{equation} 
By Galerkin approach, the displacements and virtual displacements are given by
\begin{equation} \label{u_and_du}
\textbf{u}^e = \sum_{i=1}^{n_{cp}^e} R_i \textbf{u}_i \qquad \delta\textbf{u}^e
= \sum_{i=1}^{n_{cp}^e} R_i \delta\textbf{u}_i
\end{equation}
where $\textbf{u}_i$ and $\delta\textbf{u}_i$ are values at $i$th control point.
The strain displacement matrix \textbf{B} is given by
\begin{equation} \label{BMatrix}
\textbf{B} =
\begin{bmatrix}
R_{1,x} & 0 & 0 & R_{2,x} & 0 & 0 & .... & R_{n_{cp}^e,x} & 0 & 0 \\
0 &R_{1,y} & 0 & 0 & R_{2,y} & 0 & .... & 0 & R_{n_{cp}^e,y} & 0  \\
0 & 0 & R_{1,z} &0 & 0 & R_{2,z} & .... &0 & 0 & R_{n_{cp}^e,z}  \\
R_{1,y} & R_{1,x} & 0 & R_{2,y} & R_{2,x} & 0 & .... & R_{n_{cp}^e,y} &
R_{n_{cp}^e,x} & 0 \\
0 & R_{1,z} & R_{1,y} & 0 & R_{2,z} & R_{2,y} & .... & 0 & R_{n_{cp}^e,z} &
R_{n_{cp}^e,y}\\
R_{1,z} &0 & R_{1,x} & R_{2,z} &0 & R_{2,x} & .... &R_{n_{cp}^e,z} &0
&R_{n_{cp}^e,x}
\end{bmatrix}
\end{equation}
By substituting Eqs, (\ref{u_and_du}) and (\ref{BMatrix}) in
Eq.(\ref{FinalWeakform}), the weak form in matrix terms can be written as
\begin{equation} \label{MatrixWeakForm}
\sum_{e=1}^{nel} \left[ \left( \int_{\Omega_e} \textbf{B}^T \textbf{C}
\textbf{B} d\Omega \right) \right] \textbf{u} = \int_{\Gamma^e_t}
\textbf{R}^T.\textbf{t} d\Gamma + \int_{\Omega^e_t} \textbf{R}^T.\textbf{f}
d\Omega 
\end{equation}
where \text{R} is defined as \\
for the boundary $\Gamma^e$ 
\begin{equation} \label{RMatrix1}
\textbf{R} =
\begin{bmatrix}
R_1(\xi,\eta) & 0 & R_2(\xi,\eta) &0 & .... & R_{n_{cp}^e}(\xi,\eta) & 0 \\
0 &R_1(\xi,\eta) & 0 & R_2(\xi,\eta) & .... & 0 & R_{n_{cp}^e}(\xi,\eta) \\
\end{bmatrix}
\end{equation}
for the domain $\Omega^e$
\begin{equation} \label{RMatrix2}
\textbf{R} =
\begin{bmatrix}
R_1(\xi,\eta,\zeta) & 0 & R_2(\xi,\eta,\zeta) &0 & .... &
R_{n_{cp}^e}(\xi,\eta,\zeta) & 0 \\
0 &R_1(\xi,\eta,\zeta) & 0 & R_2(\xi,\eta,\zeta) & .... & 0 &
R_{n_{cp}^e}(\xi,\eta,\zeta) \\
\end{bmatrix}
\end{equation}
Eq.(\ref{MatrixWeakForm}) can be rewritten in a standard matrix form as
\begin{equation}
\sum_{e=1}^{nel} [\textbf{K}^e \textbf{U}^e = \textbf{F}^e] 
\end{equation}
where $\textbf{K}^e$ is isogeometric element’s stiffness matrix, $\textbf{U}^e$
is displacement vector and $\textbf{F}^e$ force vector





\section{Piezoelectric Case}


\subsection{Governing Equations for Piezoelectric Materials}
The coupled electro-mechanical interactions are governed by conservation of
momentum and Gauss's law as \cite{nguyen2018finite}:
\begin{equation} \label{MechEq}
\sigma_{ij,j} + f_i = 0
\end{equation} 
\begin{equation} \label{ElecEq}
D_{i,i} - q = 0
\end{equation}
Where, $ f_i $ is body force, $ q $ is electrical charge, $ \sigma_{ij} $ is
Cauchy stress tensor and $ D_i $ is electrical displacement vector.
The constitutive equations for electromechanical coupling are defined as
\begin{equation}
\sigma_{ij} = C_{ijkl} \varepsilon_{kl} - e_{kij} E_k
\end{equation}
\begin{equation}
D_i = e_{ikl} \varepsilon_{kl} + \kappa_{ik} E_k
\end{equation}
Where, $ C $ , $ e $ and $ \kappa $ are elastic,
piezoelectric and dielectric material constants respectively. The Cauchy strain
tensor is defined as:
\begin{equation}
\varepsilon_{ij} = \frac{1}{2} (u_{i,j} + u_{j,i})
\end{equation}
and electric field vector as:
\begin{equation}
E_i = -\phi_{,i}
\end{equation}


\subsection{Weak Formulation}
Applying the principle of virtual work to the Eq.(\ref{MechEq})
and Eq. (\ref{ElecEq}) we can write \cite{nguyen2018finite}
\begin{equation} \label{CouplingWeakForm1}
\int_\Omega ( \sigma_{ij,j} + f_i ) \delta u_i d\Omega = 0  
\end{equation}
\begin{equation} \label{CouplingWeakForm2}
\int_\Omega (D_{i,i} - q ) \delta \phi d\Omega = 0
\end{equation}
with, \\
essential boundary conditions:
$u = u_o$ on $\Gamma_u$ and $\Phi=\Phi_0$ on $\Gamma_\Phi$ \\
natural boundary condition:
$\sigma_{ij}n_j = \bar{t}_j$ on $\Gamma_t$ and $D_i n_i = q_0$ on $\Gamma_q$ \\*
where $\delta u_i$ and $\delta \phi$ are virtual or arbitrary displacement and
potential fields. \\
Integrating Eq. (\ref{CouplingWeakForm1}) and Eq. (\ref{CouplingWeakForm2}) by
parts and later applying Guass divergence theorm and boundary conditions we
approach at weak form
\begin{equation} \label{FinalCouplingWeakForm1}
\int_\Omega \sigma_{ij} \delta \epsilon_{ij} d\Omega - \left[ \int_{\Gamma}
\bar{t_i} \delta u_i d\Gamma + \int_\Omega f_i \delta u_i d\Omega  \right]= 0 
\end{equation}
\begin{equation} \label{FinalCouplingWeakForm2}
\int_\Omega D_i \delta E_i d\Omega - \left[ \int_{\Gamma} Q \delta \phi d\Gamma
+ \int_\Omega q \delta \phi d\Omega \right] = 0
\end{equation}


\subsection{IGA Formulation}
By Galerkin approach, displacements, potentials and their virtual values are
given by below equations \cite{agrawal2019iga}

\begin{equation} \label{u_and_du_2}
\textbf{u}^e = \sum_{i=1}^{n_{cp}^e} R_i \textbf{u}_i \qquad \delta\textbf{u}^e
= \sum_{i=1}^{n_{cp}^e} R_i \delta\textbf{u}_i
\end{equation}

\begin{equation} \label{Phi_and_dPhi_2}
\Phi^e = \sum_{i=1}^{n_{cp}^e} R_i \Phi_i \qquad \delta\Phi^e =
\sum_{i=1}^{n_{cp}^e} R_i \delta\Phi_i
\end{equation}

\noindent
where $\textbf{u}_i$ , $\delta\textbf{u}_i$, $\delta\Phi_i$ and $\Phi_i$ are
values at $i$th control point.\\
By substituting Eqs, (\ref{u_and_du_2}) and (\ref{Phi_and_dPhi_2}) in
Eq.(\ref{FinalCouplingWeakForm1}) and Eq.(\ref{FinalCouplingWeakForm2}) the weak
form in matrix notation can be written as

\begin{equation} \label{CoupledStiffnessMatrix}
\begin{bmatrix} 
K_{MM} & K_{ME} \\
K_{EM} & K_{EE} \\ 
\end{bmatrix}
\begin{bmatrix} 
\textbf{u} \\
\phi \\ 
\end{bmatrix} = \begin{bmatrix} 
F_M \\
F_E \\ 
\end{bmatrix}
\end{equation}  
Where,
\begin{equation}
K_{MM} = \int_\Omega \textbf{B}_u^T \textbf{C} \textbf{B}_u d\Omega
\end{equation}
\begin{equation}
K_{ME} = \int_\Omega \textbf{B}_u^T \textbf{e} \textbf{B}_e d\Omega
\end{equation}
\begin{equation}
K_{EM} = \int_\Omega \textbf{B}_e^T \textbf{e}^T \textbf{B}_u d\Omega
\end{equation}
\begin{equation}
K_{EE} = - \int_\Omega \textbf{B}_e^T \kappa \textbf{B}_e d\Omega
\end{equation}
\begin{equation}
F_M = \int_\Omega \textbf{R}_u^T \textbf{f} d\Omega + \int_\Gamma \textbf{R}_u^T
\textbf{t} d\Gamma 
\end{equation}
\begin{equation}
F_E = \int_\Omega \textbf{R}_e^T q d\Omega + \int_\Gamma \textbf{R}_e^T Q
d\Gamma 
\end{equation}

\noindent
where \text{R} is defined as \\
for the boundary $\Gamma$ 

\begin{equation} \label{RMatrix11}
\textbf{R}_u =
\begin{bmatrix}
R_1(\xi,\eta) & 0 & R_2(\xi,\eta) &0 & .... & R_{n_{cp}^e}(\xi,\eta) & 0 \\
0 &R_1(\xi,\eta) & 0 & R_2(\xi,\eta) & .... & 0 & R_{n_{cp}^e}(\xi,\eta) \\
\end{bmatrix}
\end{equation}

\begin{equation} \label{RMatrix12}
\textbf{R}_e =
\begin{bmatrix}
R_1(\xi,\eta) & R_2(\xi,\eta) & .... & R_{n_{cp}^e}(\xi,\eta) \\
\end{bmatrix}
\end{equation}

\noindent
for the domain $\Omega$

\begin{equation} \label{RMatrix21}
\textbf{R}_u =
\begin{bmatrix}
R_1(\xi,\eta,\zeta) & 0 & R_2(\xi,\eta,\zeta) &0 & .... &
R_{n_{cp}^e}(\xi,\eta,\zeta) & 0 \\
0 &R_1(\xi,\eta,\zeta) & 0 & R_2(\xi,\eta,\zeta) & .... & 0 &
R_{n_{cp}^e}(\xi,\eta,\zeta) \\
\end{bmatrix}
\end{equation}

\begin{equation} \label{RMatrix22}
\textbf{R}_e =
\begin{bmatrix}
R_1(\xi,\eta,\zeta) & R_2(\xi,\eta,\zeta) & .... & R_{n_{cp}^e}(\xi,\eta,\zeta)
\\
\end{bmatrix}
\end{equation}

\noindent
B matix is given as

\begin{equation} \label{BuMatrix}
\textbf{B}_u =
\begin{bmatrix}
R_{1,x} & 0 & 0 & R_{2,x} & 0 & 0 & .... & R_{n_{cp}^e,x} & 0 & 0 \\
0 &R_{1,y} & 0 & 0 & R_{2,y} & 0 & .... & 0 & R_{n_{cp}^e,y} & 0  \\
0 & 0 & R_{1,z} &0 & 0 & R_{2,z} & .... &0 & 0 & R_{n_{cp}^e,z}  \\
R_{1,y} & R_{1,x} & 0 & R_{2,y} & R_{2,x} & 0 & .... & R_{n_{cp}^e,y} &
R_{n_{cp}^e,x} & 0 \\
0 & R_{1,z} & R_{1,y} & 0 & R_{2,z} & R_{2,y} & .... & 0 & R_{n_{cp}^e,z} &
R_{n_{cp}^e,y}\\
R_{1,z} &0 & R_{1,x} & R_{2,z} &0 & R_{2,x} & .... &R_{n_{cp}^e,z} &0
&R_{n_{cp}^e,x}
\end{bmatrix}
\end{equation}

\begin{equation} \label{BeMatrix}
\textbf{B}_e =
\begin{bmatrix}
R_{1,x} & R_{2,x} & .... & R_{n_{cp}^e,x} \\
R_{1,y} & R_{2,y} & .... & R_{n_{cp}^e,y} \\
R_{1,z} & R_{2,z} & .... & R_{n_{cp}^e,z} \\
\end{bmatrix}
\end{equation}

\begin{equation}
\varepsilon = \textbf{B}_u . \textbf{u}
\end{equation}

\begin{equation}
\textbf{E} = - \textbf{B}_e . \Phi
\end{equation}

\noindent
Eq. (\ref{CoupledStiffnessMatrix}) can be solved using numerical methods like
Newton Raphson scheme for displacements and potential solution field.

\section{Modelling and Results}
\subsection{2D Plate with linear elastic loading}
\subsubsection{Problem description}
A 2D plate is subjected to mechanical loading as shown in Figure(\ref{XYLoading}).The material used is PZT-PIC151 ceramics \cite{kozinov2018simulation} and the elastic constants of the material are given in appendix (\ref{MaterialProps}).
The movement of bottom edge AB is fixed in y direction and left edge AC in x direction. A displacement of 0.1 mm on right edge BD and 0.2 mm on top edge CD in x and y directions are given respectively.
\begin{figure}[H]
	\centering
	\begin{minipage}{.5\textwidth}
		\centering
		\includegraphics[width=0.8\linewidth]{2DPlate.png}
		\captionof{figure}{2D Plate}
		\label{2Dplate}
	\end{minipage}%
	\begin{minipage}{.5\textwidth}
		\centering
		\includegraphics[width=0.9\linewidth]{XYLoading.png}
		\captionof{figure}{2D Plate with loading}
		\label{XYLoading}
	\end{minipage}
\end{figure}
\subsubsection{Parametric details for the plate with single element}
\label{ParametricDeatils1Elem}
The 2nd order NURBS curve is used in both $\xi$ and $\eta$ directions. \\
\begin{comment}
The knot vectors along $\xi$ and $\eta$ directions are \\
$\Xi= [0,0,1,1]$ and $\eta= [0,0,1,1]$. \\
Control points along $\xi$ direction is given by \\
$n_{cp}(\xi)$ = total number of knots in $[\Xi] - (p+1) = 2$.\\
Similarly the total number of control points along $\eta$ direction is given
by\\ $n_{cp}(\eta)$ = total number of knots in $[H ]- (q+1) = 2$ . \\
The total number of control points which defines the surface is\\
$n_{cp}$ = $n_{cp}(\xi) * n_{cp}(\eta)$ which is $2*2 = 4$. \\
\end{comment}
%\begin{verbatim}
\begin{enumerate}
	\item Physical details for the geometry: \\
	L = 10 \qquad \# Length of the plate in mm \\
	H = 10 \qquad \# Height of the plate in mm \\
	T =  1 {}  \qquad \# Thickness of the plate in mm \\
	
	\item Parametric details of the geometry: \\
	$\Xi$ = [0,0,1,1] \qquad \# Knot vector in $\xi$  direction \\
	$H$ = [0,0,1,1] \qquad \# Knot vector in $\eta$ direction \\
	
	Degree of the curve \\
	p=1 \qquad \# Degree of the curve in $\xi$  direction \\
	q=1 \qquad \# Degree of the curve in $\eta$ direction \\
	
	Number of control points in each direction \\
	$n_{cp}^{\xi}$  = len($\Xi$) - (p+1)  \qquad \#No.of control points in $\xi$
	direction (4-(1+1) = 2) \\
	$n_{cp}^{\eta}$ = len($H$) - (q+1)  \qquad \#No.of control points in $\eta$
	direction (4-(1+1) = 2) \\
	
	\item Total number of control points for the geometry \\
	$n_{cp} = n_{cp}^{\xi} * n_{cp}^{\eta}$ = 2*2 = 4\\
	\begin{comment}
	\item Control points net for the geometry \\
	P = [[[0,0,0,1],[L,0,0,1]],
	[[0,L,0,1],[L,L,0,1]]]
	
	$$
	P = \begin{bmatrix}
	[0,0,0,1] & [L,0,0,1] \\
	[L,0,0,1] & [L,L,0,1] \\
	\end{bmatrix} 
	$$
	
	\end{enumerate}
	%\end{verbatim}
	\end{comment}
	The control points are given by
	\begin{center}
		\begin{tabular}{ |c|c|c|c|c| } 
			\hline
			i & $ P_{i,0} $ & $ P_{i,1} $  \\ \hline
			0 & $ (0,0,0,1) $ & $ (0,10,0,1) $  \\ \hline
			1 & $ (10,0,0,1) $ & $ (10,10,0,1) $  \\ \hline
			
		\end{tabular}
	\end{center}
	with fourth value in the paranthesis being weights of respective control points.
	**As this is the case of single element, there is no need for the control point
	assembly array and knot vector connectivity matrix.
\end{enumerate}
\subsubsection{Results and discussions}
Considering the accuracy of the IGA simulation results over FEM results, IGA code generated results can be compared with abaqus inbuilt element generated results. An abaqus plane strain full integration element (\textbf{CPE4}) \cite{abaqus10version} is used for this purpose.
\\The below figures shows the values of displacements (U) and reaction forces (RF ) for both abaqus and IGA element.\\

\textbf{**A similar contour is used for the program generated results and abaqus results for an easy comparison. }\\
Figure(\ref{M1U1}) and Figure(\ref{M1U1_IGA}) shows the displacement (U1) values of the single CPE4 element and single IGA element at 100 \% loading in x direction respectively. \\
\begin{figure}[H]
	\centering
	\begin{minipage}{.5\textwidth}
		\centering
		\includegraphics[width=1\linewidth]{M1U1.png}
		\captionof{figure}{CPE4 Element:U1 \\\textbf{Abaqus generated result}}
		\label{M1U1}
	\end{minipage}%
	\begin{minipage}{.5\textwidth}
		\centering
		\includegraphics[width=1\linewidth]{M1U1_IGA.png}
		\captionof{figure}{IGA Element:U1 \\ \textbf{Program generated result}}
		\label{M1U1_IGA}
	\end{minipage}
\end{figure}
\begin{comment}
		\begin{figure}[H]
		\begin{center}
		\includegraphics[scale=0.45]{xyz.png} 
		\caption{\\CPE4 Element U1}\label{xyz}
		\end{center}	
		\end{figure}
		
		\begin{figure}[H]
		\begin{center}
		\includegraphics[scale=0.8]{Figure_1.png} 
		\caption{\\IGA Element U1}\label{Figure_1}
		\end{center}	
		\end{figure}
\end{comment}
Figure(\ref{M1U2}) and Figure(\ref{M1U2_IGA}) shows the displacement (U2) values of the single CPE4 element and single IGA element at 100 \% loading in y direction respectively. \\
\begin{figure}[H]
	\centering
	\begin{minipage}{.5\textwidth}
		\centering
		\includegraphics[width=1\linewidth]{M1U2.png}
		\captionof{figure}{CPE4 Element:U2
		\\\textbf{Abaqus generated result}}
		\label{M1U2}
	\end{minipage}%
	\begin{minipage}{.5\textwidth}
		\centering
		\includegraphics[width=1\linewidth]{M1U2_IGA.png}
		\captionof{figure}{IGA Element:U2\\ \textbf{Program generated result}}
		\label{M1U2_IGA}
	\end{minipage}
\end{figure}



Figure(\ref{M1RF1}) and Figure(\ref{M1RF1_IGA}) shows the Reaction force values (RF1) of the single CPE4 element and single IGA element at 100 \% loading in x direction respectively. \\
\begin{figure}[H]
	\centering
	\begin{minipage}{.5\textwidth}
		\centering
		\includegraphics[width=1\linewidth]{M1RF1.png}
		\captionof{figure}{CPE4 Element:RF1
		\\\textbf{Abaqus generated result}}
		\label{M1RF1}
	\end{minipage}%
	\begin{minipage}{.5\textwidth}
		\centering
		\includegraphics[width=1\linewidth]{M1RF1_IGA.png}
		\captionof{figure}{IGA Element:RF1\\ \textbf{Program generated result}}
		\label{M1RF1_IGA}
	\end{minipage}
\end{figure}
Figure(\ref{M1RF2}) and Figure(\ref{M1RF2_IGA}) shows the Reaction force values (RF2) of the single CPE4 element and single IGA element at 100 \% loading in y direction respectively. \\
\begin{figure}[H]
	\centering
	\begin{minipage}{.5\textwidth}
		\centering
		\includegraphics[width=1\linewidth]{M1RF2.png}
		\captionof{figure}{CPE4 Element:RF2
		\\\textbf{Abaqus generated result}}
		\label{M1RF2}
	\end{minipage}%
	\begin{minipage}{.5\textwidth}
		\centering
		\includegraphics[width=1\linewidth]{M1RF2_IGA.png}
		\captionof{figure}{IGA Element:RF2\\ \textbf{Program generated result}}
		\label{M1RF2_IGA}
	\end{minipage}
\end{figure}
\subsubsection{Conclusion}
As shown in figures above the values generated by IGA code are inline with the results of Abaqus element. So it can be concluded that IGA code written works well with 2D one element case.






\subsection{2D Plate with pure Electrical loading}
\subsubsection{Problem description}
A 2D plate is subjected to Electrical loading as shown in Figure(\ref{PureElectrical}).The material used is PZT-PIC151 ceramics and the dielectric constants can be seen in Appendix (\ref{MaterialProps})
The movement of bottom edge AB is fixed in y direction and left edge AC in x direction. An electrical potential V of 1000 volts is applied at one node D as shown in Fig.(\ref{PureElectrical})
\begin{figure}[H]
	\centering
	\begin{minipage}{.5\textwidth}
		\centering
		\includegraphics[width=0.8\linewidth]{2DPlate.png}
		\captionof{figure}{2D Plate}
		\label{2Dplate}
	\end{minipage}%
	\begin{minipage}{.5\textwidth}
		\centering
		\includegraphics[width=1\linewidth]{PureElectrical.png}
		\captionof{figure}{2D Plate with pure electrical loading}
		\label{PureElectrical}
	\end{minipage}
\end{figure}
\subsubsection{Parametric details for the plate with single element}


The parametric details for the geometry are the same as in section \hyperref[ParametricDeatils1Elem]{\ref{ParametricDeatils1Elem}}

\subsubsection{Results and discussions}
In this section the comparison is made between IGA code generated result and abaqus plane strain full integration piezoelectric element (\textbf{CPE4E}) \cite{abaqus10version}.\\The below figures shows the values of Electrical potentials (EPOT) and reactive electrical nodal charge (RCHG) for both abaqus and IGA element.\\
\\\textbf{*** Electro-Mechanical coupling is deactivated in this case by giving all the piezoelectric constants a value of "zero"  }
\\
\textbf{***A similar contour is used for the program generated results and abaqus results for an easy comparison. }\\
\\
Figure(\ref{E1EPOT}) and Figure(\ref{E1EPOT_IGA}) shows the Electrical potential (EPOT) values of the single CPE4E element and single IGA element at 100 \% loading respectively. \\
\begin{figure}[H]
	\centering
	\begin{minipage}{.5\textwidth}
		\centering
		\includegraphics[width=1\linewidth]{E1EPOT.png}
		\captionof{figure}{CPE4E Element:EPOT \\\textbf{Abaqus generated result}}
		\label{E1EPOT}
	\end{minipage}%
	\begin{minipage}{.5\textwidth}
		\centering
		\includegraphics[width=1\linewidth]{E1EPOT_IGA.png}
		\captionof{figure}{IGA Element:EPOT \\ \textbf{Program generated result}}
		\label{E1EPOT_IGA}
	\end{minipage}
\end{figure}
\begin{comment}
\begin{figure}[H]
\begin{center}
\includegraphics[scale=0.45]{xyz.png} 
\caption{\\CPE4 Element U1}\label{xyz}
\end{center}	
\end{figure}

\begin{figure}[H]
\begin{center}
\includegraphics[scale=0.8]{Figure_1.png} 
\caption{\\IGA Element U1}\label{Figure_1}
\end{center}	
\end{figure}
\end{comment}
Figure(\ref{E1RCHG}) and Figure(\ref{E1RCHG_IGA}) shows the reactive nodal charge (RCHG) values of the single CPE4E element and single IGA element at 100 \% loading respectively. \\
\begin{figure}[H]
	\centering
	\begin{minipage}{.5\textwidth}
		\centering
		\includegraphics[width=1\linewidth]{E1RCHG.png}
		\captionof{figure}{CPE4 Element:RCHG
		\\\textbf{Abaqus generated result}}
		\label{E1RCHG}
	\end{minipage}%
	\begin{minipage}{.5\textwidth}
		\centering
		\includegraphics[width=1\linewidth]{E1RCHG_IGA.png}
		\captionof{figure}{IGA Element:RCHG \\ \textbf{Program generated result}}
		\label{E1RCHG_IGA}
	\end{minipage}
\end{figure}

\subsubsection{Conclusion}
As shown in figures above the values generated by IGA code are inline with the results of Abaqus element. So it can be concluded that for electrical analysis IGA code written works well with 2D one element case.



\subsection{2D Piezoelectric plate}
\subsubsection{Problem description}
A 2D piezoelectric plate subjected to mechanical displacements and electrical
loading is considered as shown in Fig.(\ref{EMLoading}) . The material used is PZT-PIC151 ceramics and the properties can be seen in Appendix (\ref{MaterialProps})

\begin{figure}[H]
	\centering
	\begin{minipage}{.5\textwidth}
		\centering
		\includegraphics[width=0.8\linewidth]{2DPlate.png}
		\captionof{figure}{2D Piezoelectric Plate}
		\label{2Dplate}
	\end{minipage}%
	\begin{minipage}{.5\textwidth}
		\centering
		\includegraphics[width=1\linewidth]{Grounded.png}
		\captionof{figure}{2D Piezoelectric Plate with loading}
		\label{EMLoading}
	\end{minipage}
\end{figure}
The movement of the bottom edge AB and left edge AC of 2D piezoelectric plate is fixed
in y direction and x direction respectively as shown in figure(\ref{EMLoading}).
The left edge AC is grounded (Electric potential $\Phi = 0$)and a displacement load
of 0.1 mm and 0.2 mm is applied on the right edge BD and top edge CD respectively. The
results for single element case and multiple elements is discussed in the below
sections. \\
The results generated by IGA code is compared with inbuilt Abaqus piezoelectric
element \textbf{CPE4E}

\subsubsection{Parametric details for the plate with single element}


The parametric details for the geometry are the same as in section \hyperref[ParametricDeatils1Elem]{\ref{ParametricDeatils1Elem}}

\subsubsection{Results and discussions}
Abaqus plane strain full integration piezoelectric element (\textbf{CPE4E}) is used for analysis. The below figures shows the values of displacements (U), electrical potentials (EPOT) and reaction forces (RF ) for both abaqus and IGA element.



Figure(\ref{EM1U1}) and Figure(\ref{EM1U1_IGA}) shows the displacement (U1) values of the single CPE4E element and single IGA element at 100 \% loading in x direction respectively. \\
\begin{figure}[H]
	\centering
	\begin{minipage}{.5\textwidth}
		\centering
		\includegraphics[width=1\linewidth]{EM1U1.png}
		\captionof{figure}{CPE4E Element:U1
		\\\textbf{Abaqus generated result}}
		\label{EM1U1}
	\end{minipage}%
	\begin{minipage}{.5\textwidth}
		\centering
		\includegraphics[width=1\linewidth]{EM1U1_IGA.png}
		\captionof{figure}{IGA Piezoelectric Element:U1\\ \textbf{Program generated result}}
		\label{EM1U1_IGA}
	\end{minipage}
\end{figure}
\noindent
Figure(\ref{EM1U2}) and Figure(\ref{EM1U2_IGA}) shows the displacement (U2) values of the single CPE4E element and single IGA element at 100 \% loading in y direction respectively. \\
\begin{figure}[H]
	\centering
	\begin{minipage}{.5\textwidth}
		\centering
		\includegraphics[width=1\linewidth]{EM1U2.png}
		\captionof{figure}{CPE4E Element:U2
		\\\textbf{Abaqus generated result}}
		\label{EM1U2}
	\end{minipage}%
	\begin{minipage}{.5\textwidth}
		\centering
		\includegraphics[width=1\linewidth]{EM1U2_IGA.png}
		\captionof{figure}{IGA Piezoelectric Element:U2\\ \textbf{Program generated result}}
		\label{EM1U2_IGA}
	\end{minipage}
\end{figure}
	\begin{comment}
\begin{figure}[H]
	\begin{center}
		\includegraphics[scale=0.45]{xyz.png} 
		\caption{\\CPE4 Element U1}\label{xyz}
	\end{center}	
\end{figure}

\begin{figure}[H]
	\begin{center}
		\includegraphics[scale=0.8]{Figure_1.png} 
		\caption{\\IGA Element U1}\label{Figure_1}
	\end{center}	
\end{figure}
	\end{comment}
	


Figure(\ref{EM1RF1}) and Figure(\ref{EM1RF1_IGA}) shows the Reaction force values (RF1) of the single CPE4E element and single IGA element at 100 \% loading in x direction respectively. \\
\begin{figure}[H]
	\centering
	\begin{minipage}{.5\textwidth}
		\centering
		\includegraphics[width=1\linewidth]{EM1RF1.png}
		\captionof{figure}{CPE4E Element:RF1
		\\\textbf{Abaqus generated result}}
		\label{EM1RF1}
	\end{minipage}%
	\begin{minipage}{.5\textwidth}
		\centering
		\includegraphics[width=1\linewidth]{EM1RF1_IGA.png}
		\captionof{figure}{IGA Piezoelectric Element:RF1\\ \textbf{Program generated result}}
		\label{EM1RF1_IGA}
	\end{minipage}
\end{figure}
Figure(\ref{EM1RF2}) and Figure(\ref{EM1RF2_IGA}) shows the Reaction force values (RF2) of the single CPE4E element and single IGA element at 100 \% loading in y direction respectively. \\
\begin{figure}[H]
	\centering
	\begin{minipage}{.5\textwidth}
		\centering
		\includegraphics[width=1\linewidth]{EM1RF2.png}
		\captionof{figure}{CPE4E Element:RF2
		\\\textbf{Abaqus generated result}}
		\label{EM1RF2}
	\end{minipage}%
	\begin{minipage}{.5\textwidth}
		\centering
		\includegraphics[width=1\linewidth]{EM1RF2_IGA.png}
		\captionof{figure}{IGA Piezoelectric Element:RF2\\ \textbf{Program generated result}}
		\label{EM1RF2_IGA}
	\end{minipage}
\end{figure}



Figure(\ref{EM1EPOT}) and Figure(\ref{EM1EPOT_IGA}) shows the Electrical potential values (EPOT) of the single CPE4E element and single IGA element at 100 \% loading respectively. \\
\begin{figure}[H]
	\centering
	\begin{minipage}{.5\textwidth}
		\centering
		\includegraphics[width=1\linewidth]{EM1EPOT.png}
		\captionof{figure}{CPE4E Element:EPOT
		\\\textbf{Abaqus generated result}}
		\label{EM1EPOT}
	\end{minipage}%
	\begin{minipage}{.5\textwidth}
		\centering
		\includegraphics[width=1\linewidth]{EM1EPOT_IGA.png}
		\captionof{figure}{IGA Piezoelectric Element:EPOT\\ \textbf{Program generated result}}
		\label{EM1EPOT_IGA}
	\end{minipage}
\end{figure}
\subsubsection{Conclusion}
As shown in figures above the values generated by IGA code for an electro-mechanical coupling are inline with the results of Abaqus element. So it can be concluded that IGA code written works well with a 2D one element case.

\subsubsection{Parametric details for the plate with 2 elements in x direction and  3 elements in y direction}
The 2nd order NURBS curve is used in both $\xi$ and $\eta$ directions. \\
\begin{comment}
The knot vectors along $\xi$ and $\eta$ directions are \\
$\Xi= [0,0,1,1]$ and $\eta= [0,0,1,1]$. \\
Control points along $\xi$ direction is given by \\
$n_{cp}(\xi)$ = total number of knots in $[\Xi] - (p+1) = 2$.\\
Similarly the total number of control points along $\eta$ direction is given
by\\ $n_{cp}(\eta)$ = total number of knots in $[H ]- (q+1) = 2$ . \\
The total number of control points which defines the surface is\\
$n_{cp}$ = $n_{cp}(\xi) * n_{cp}(\eta)$ which is $2*2 = 4$. \\
\end{comment}
\begin{comment}
\item Control points net for the geometry \\
P = [[[0,0,0,1],[L,0,0,1]],
[[0,L,0,1],[L,L,0,1]]]

$$
P = \begin{bmatrix}
[0,0,0,1] & [L,0,0,1] \\
[L,0,0,1] & [L,L,0,1] \\
\end{bmatrix} 
$$

\end{enumerate}
%\end{verbatim}
\end{comment}
%\begin{verbatim}
\begin{enumerate}
	\item Physical details for the geometry: \\
	L = 10 \qquad \# Length of the plate in mm \\
	H = 10 \qquad \# Height of the plate in mm \\
	T = 1 {} \qquad \# Thickness of the plate in mm \\
	
	\item Parametric details of the geometry: \\
	$\Xi$ =  [0,0,1,2,2] \qquad \# Knot vector in xi  direction \\
	$H$ = [0,0,1,2,3,3] \qquad \# Knot vector in eta direction \\
	
	Degree of the curve \\
	p=1 \qquad \# Degree of the curve in $\xi$  direction \\
	q=1 \qquad \# Degree of the curve in $\eta$ direction \\
	
	Number of control points in each direction \\
	$n_{cp}^{\xi}$  = len($\Xi$) - (p+1)  \qquad \#No.of control points in $\xi$
	direction (5-(1+1) = 3) \\
	$n_{cp}^{\eta}$ = len($H$) - (q+1)  \qquad \#No.of control points in $\eta$
	direction (6-(1+1) = 4) \\
	
	\item Total number of control points for the geometry \\
	$n_{cp} = n_{cp}^{\xi} * n_{cp}^{\eta}$ = 3*4 = 12\\
	The control points are given by
	\begin{center}
		\begin{tabular}{ |c|c|c|c|c|c| } 
			\hline
			i & $ P_{i,0} $ & $ P_{i,1} $ & $ P_{i,2} $   \\ \hline
			0 & $ (0,0,0,1) $ & $ (0,5,0,1) $ & $ (0,10,0,1) $  \\ \hline
			1 & $ (3.33,0,0,1) $ & $ (3.33,5,0,1) $ & $ (3.33,10,0,1) $  \\ \hline
			2 & $ (6.67,0,0,1) $ & $ (6.67,5,0,1) $ & $ (6.67,10,0,1) $  \\ \hline
			3 & $ (10,0,0,1) $ & $ (10,5,0,1) $ & $ (10,10,0,1) $  \\ \hline
		\end{tabular}
	\end{center}
	with fourth value in the paranthesis being weights of respective control points.
	A control point assembly array and knot vector connectivity matrix are used to connect the elements.
\end{enumerate}

\subsubsection{Results and discussions}
Abaqus plane strain full integration piezoelectric element (\textbf{CPE4E}) is used for analysis. For the comparision between abaqus and IGA elements 2 elements, along x direction and 3 elements along y direction are used. A different number of elements are used along each direction in order to verify if the code generates proper results in unsymmetric conditions aswell w.r.t number of elements in each direction.\\
 The below figures shows the values of displacements (U), electrical potentials (EPOT) and reaction forces (RF ) for both abaqus and IGA elements.\\
Figure(\ref{EM23U1}) and Figure(\ref{EM23U1_IGA}) shows the displacement (U1) values of the CPE4E elements and IGA elements at 100 \% loading in x direction respectively. \\
\begin{figure}[H]
	\centering
	\begin{minipage}{.5\textwidth}
		\centering
		\includegraphics[width=1\linewidth]{EM23U1.png}
		\captionof{figure}{CPE4E Element:U1
		\\\textbf{Abaqus generated result}}
		\label{EM23U1}
	\end{minipage}%
	\begin{minipage}{.5\textwidth}
		\centering
		\includegraphics[width=1\linewidth]{EM23U1_IGA.png}
		\captionof{figure}{IGA Piezoelectric Element:U1\\ \textbf{Program generated result}}
		\label{EM23U1_IGA}
	\end{minipage}
\end{figure}
\noindent
Figure(\ref{EM23U2}) and Figure(\ref{EM23U2_IGA}) shows the displacement (U2) values of the CPE4E elements and single IGA elements at 100 \% loading in y direction respectively. \\
\begin{comment}
\begin{figure}[H]
\begin{center}
\includegraphics[scale=0.45]{xyz.png} 
\caption{\\CPE4 Element U1}\label{xyz}
\end{center}	
\end{figure}

\begin{figure}[H]
\begin{center}
\includegraphics[scale=0.8]{Figure_1.png} 
\caption{\\IGA Element U1}\label{Figure_1}
\end{center}	
\end{figure}
\end{comment}
\begin{figure}[H]
	\centering
	\begin{minipage}{.5\textwidth}
		\centering
		\includegraphics[width=1\linewidth]{EM23U2.png}
		\captionof{figure}{CPE4E Element:U2
		\\\textbf{Abaqus generated result}}
		\label{EM23U2}
	\end{minipage}%
	\begin{minipage}{.5\textwidth}
		\centering
		\includegraphics[width=1\linewidth]{EM23U2_IGA.png}
		\captionof{figure}{IGA Piezoelectric Element:U2\\ \textbf{Program generated result}}
		\label{EM23U2_IGA}
	\end{minipage}
\end{figure}
Figure(\ref{EM23RF1}) and Figure(\ref{EM23RF1_IGA}) shows the Reaction force values (RF1) of the CPE4E elements and IGA elements at 100 \% loading in x direction respectively. \\
\begin{figure}[H]
	\centering
	\begin{minipage}{.5\textwidth}
		\centering
		\includegraphics[width=1\linewidth]{EM23RF1.png}
		\captionof{figure}{CPE4E Element:RF1
		\\\textbf{Abaqus generated result}}
		\label{EM23RF1}
	\end{minipage}%
	\begin{minipage}{.5\textwidth}
		\centering
		\includegraphics[width=1\linewidth]{EM23RF1_IGA.png}
		\captionof{figure}{IGA Piezoelectric Element:RF1\\ \textbf{Program generated result}}
		\label{EM23RF1_IGA}
	\end{minipage}
\end{figure}
Figure(\ref{EM23RF2}) and Figure(\ref{EM23RF2_IGA}) shows the Reaction force values (RF2) of the CPE4E elements and IGA elements at 100 \% loading in y direction respectively. \\
\begin{figure}[H]
	\centering
	\begin{minipage}{.5\textwidth}
		\centering
		\includegraphics[width=1\linewidth]{EM23RF2.png}
		\captionof{figure}{CPE4E Element:RF2
		\\\textbf{Abaqus generated result}}
		\label{EM23RF2}
	\end{minipage}%
	\begin{minipage}{.5\textwidth}
		\centering
		\includegraphics[width=1\linewidth]{EM23RF2_IGA.png}
		\captionof{figure}{IGA Piezoelectric Element:RF2\\ \textbf{Program generated result}}
		\label{EM23RF2_IGA}
	\end{minipage}
\end{figure}
Figure(\ref{EM23EPOT}) and Figure(\ref{EM23EPOT_IGA}) shows the Electrical potential values (EPOT) of the CPE4E elements and IGA elements at 100 \% loading respectively. \\
\begin{figure}[H]
	\centering
	\begin{minipage}{.5\textwidth}
		\centering
		\includegraphics[width=1\linewidth]{EM23EPOT.png}
		\captionof{figure}{CPE4E Element:EPOT
		\\\textbf{Abaqus generated result}}
		\label{EM23EPOT}
	\end{minipage}%
	\begin{minipage}{.5\textwidth}
		\centering
		\includegraphics[width=1\linewidth]{EM23EPOT_IGA.png}
		\captionof{figure}{IGA Piezoelectric Element:EPOT\\ \textbf{Program generated result}}
		\label{EM23EPOT_IGA}
	\end{minipage}
\end{figure}

\subsubsection{Conclusion}
As shown in figures above for a 6 elements case the results generated by IGA code are inline with the results of the Abaqus elements.   




\section{Milestones achieved}
The following table describes the proposed coding activities and achieved activities.


\begin{center}
	\begin{tabular}{ |c|c|c|c| } 
		\hline
		Proposed Activities & Achieved \\
		\hline
		Implementation of Isogeometric Analysis & yes \\ 
		IGA implementation for a Single element 2D case& yes \\ 
		Coupling between mechanical and electrical Dof’s & yes \\ 
		Verification of results with abaqus inbuilt piezoelectric element & yes \\ 
		\hline
		Extra Activities & Achieved \\
		\hline
		Implementation of Knot and Control point assembly arrays & yes \\
		Code extension to multiple elements & yes \\
		\hline
	\end{tabular}
\end{center}

\section{Appendices}

\subsection{Material Properties} \label{MaterialProps}
The material used is
PZT-PIC151 ceramics and the properties \cite{kozinov2018simulation} are as follow. The elastic constants (C) are transversely isotropic, dielectric constants ($\kappa$) are anisotropic and material is polarized in Y-direction. 

$$
C = \begin{bmatrix}
139000 & 74280 & 77840 & 0 & 0 & 0\\
47280 & 115400 & 74280 & 0 & 0 & 0\\
77840 & 74280 & 139000 & 0 & 0 & 0\\
0 & 0 & 0 & 25640 & 0 & 0\\
0 & 0 & 0 & 0 & 25640 & 0\\
0 & 0 & 0 & 0 & 0 & 25640\\
\end{bmatrix} MPa
$$  
$$
e = \begin{bmatrix}
0 & -5.20710E-6 & 0 \\
0 & 15.08E-6 & 0 \\
0 & -5.207E-6 & 0\\
12.710E-6 & 0 & 0 \\
0 & 0 & 0 \\
0 & 0 & 12.710E-6 \\
\end{bmatrix} C/ mm`2
$$
$$
\varepsilon = \begin{bmatrix}
6.752E-12 & 0 & 0 \\
0 & 5.872E-12 & 0 \\
0 & 0 & 6.752E-12\\
\end{bmatrix} C/(V mm)
$$
where \\
e is Piezoelectric constants.








\noindent
%*** Remember References and citations ***
\bibliographystyle{plain}
\bibliography{References} 
\end{document}

